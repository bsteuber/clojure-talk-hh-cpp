% Created 2017-06-14 Wed 19:36
\documentclass[presentation]{beamer}
\usepackage[utf8]{inputenc}
\usepackage[T1]{fontenc}
\usepackage{fixltx2e}
\usepackage{graphicx}
\usepackage{longtable}
\usepackage{float}
\usepackage{wrapfig}
\usepackage{rotating}
\usepackage[normalem]{ulem}
\usepackage{amsmath}
\usepackage{listings}
\usepackage{textcomp}
\usepackage{marvosym}
\usepackage{wasysym}
\usepackage{amssymb}
\usepackage{hyperref}
\tolerance=1000
\usetheme{Warsaw}
\author{Benjamin Teuber}
\def\cpp{C++}
\date{\today}
\title{Clojure-Einführung}
\institute{\cpp{}-Meetup Hamburg}
\begin{document}

\maketitle
\begin{frame}{Outline}
\tableofcontents
\end{frame}



\section{Clojure-Einführung}

\begin{frame}{Grundsätzliches über Clojure}
  Eigenschaften von Clojure
  \begin{itemize}
  \item Funktionale Sprache
  \item Dynamisches Typsystem
  \item Lisp-Syntax
  \item Fokus auf Immutability
  \end{itemize}
\end{frame}

\begin{frame}{Clojure und ClojureScript}
  \begin{block}{Clojure}
    \begin{itemize}
    \item Läuft auf der JVM
    \item Kompiliert in Java-Bytecode
    \item Sehr gute Java Interop
    \end{itemize}
  \end{block}

  \begin{block}{ClojureScript}
    \begin{itemize}
    \item Kompiliert nach JavaScript
    \item Gut geeignet für Frontend-Programmierung
    \item Fast komplett kompatibel zu Clojure
    \item Direkte JS Interop
    \end{itemize}
  \end{block}
\end{frame}

\begin{frame}{Dynamische Typisierung}
  Vorteile:
  \begin{itemize}
  \item Viel Flexibilität
    \begin{itemize}
    \item Es gibt Konstrukte die schwer oder gar nicht statisch zu typisieren sind
    \end{itemize}
  \item Wenig Tippaufwand
  \end{itemize}

  Nachteile:
  \begin{itemize}
  \item Typfehler treten erst zur Laufzeit auf
    \begin{itemize}
    \item Editor kann nicht bereits Typprobleme anzeigen
    \item Unit Tests helfen hier
    \end{itemize}
  \item Performance-Einbußen
  \end{itemize}
\end{frame}

\begin{frame}{Performance}
  \begin{itemize}
  \item Generell: Programmer Time $>$ Running Time
  \item ca. Faktor 2-3 langsamer (TODO) als C++
  \end{itemize}
\end{frame}

\begin{frame}[fragile]{Clojure-Syntax}
  \begin{columns}
    \begin{column}{0.5\textwidth}
      \begin{block}{\cpp}
\begin{verbatim}
my_function(x, y)

1 + 2 + 3

int my_add(int x, int y) {
    return x + y;
}
\end{verbatim}
      \end{block}
    \end{column}
    \begin{column}{0.5\textwidth}
      \begin{block}{Clojure}
\begin{verbatim}
(my-function x y)

(+ 1 2 3)

(defn my-add [x y]
  (+ x y))

\end{verbatim}
      \end{block}
    \end{column}
  \end{columns}
\end{frame}

\begin{frame}[fragile]{Datenstrukturen}
  \begin{columns}
    \begin{column}{0.5\textwidth}
      \begin{block}{Strings}
\begin{verbatim}
(def hello
  "Hallo, Welt")
\end{verbatim}
      \end{block}
      \begin{block}{Listen}
\begin{verbatim}
(def num-list
   (list 1 2 3 4 5))
\end{verbatim}
      \end{block}
      \begin{block}{Maps}
\begin{verbatim}
(def walter
  {:first-name   "Walter"
   :familiy-name "White"}
\end{verbatim}
      \end{block}
    \end{column}
    \begin{column}{0.5\textwidth}
      \begin{block}{Keywords}
\begin{verbatim}
(def key
  :first-name)
\end{verbatim}
      \end{block}
      \begin{block}{Vektoren}
\begin{verbatim}
(def nums-vec
  [1 2 3 4])
\end{verbatim}
      \end{block}
      \begin{block}{Sets}
\begin{verbatim}
(def odd-set
  #{1 3 5 7 9})
\end{verbatim}
      \end{block}
    \end{column}
  \end{columns}
\end{frame}

\begin{frame}[fragile]{Datenstrukturen als Funktionen}
  \begin{itemize}
  \item Viele Datenstrukturen können direkt als Funktionen benutzt werden
  \item Praktisch für Higher Order Funktions wie map
  \end{itemize}
  \begin{block}{}
\begin{verbatim}
  (get nums-vec 1)                 ; => 2
  (nums-vec 1)                     ; => 2

  (nums-vec 42)                    ; => Exception

  (get walter :first-name)         ; => "Walter"
  (walter :first-name)             ; => "Walter"
  (:first-name walter)             ; => "Walter"

  (walter :middle-name)            ; => nil
  (walter :middle-name :not-found) ; => nil
\end{verbatim}

  \end{block}
\end{frame}

\begin{frame}[fragile]{Immutable Data Structures}
  \begin{itemize}
  \item Alle Datenstrukturen sind immutable
  \item Operationen liefern neue Collections zurück
  \end{itemize}
  \begin{block}{}
\begin{verbatim}
(def v [1 2 3])

(conj v 4 5 6)            ; => [1 2 3 4 5 6]

v                         ; => [1 2 3]
\end{verbatim}
  \end{block}
  \begin{itemize}
  \item Effiziente Baum-Implementierung
  \item Keine vollständige Kopie nötig
  \item ``$O(log_{32}(n))$ is good enough''
  \end{itemize}
\end{frame}

\begin{frame}{Warum Immutability?}
  \begin{itemize}
  \item ``Referenzielle Transparenz'' von Funktionen
    \begin{itemize}
    \item Verhalten komplett durch Ein- und Ausgabe beschrieben
    \item Gute Modularität
    \item Einfach zu Testen
    \end{itemize}
  \item Mehr State $=>$ Mehr Kopplung von Modulen $=>$ mehr Bugs
  \end{itemize}
\end{frame}

\begin{frame}[fragile]{Stream-Verarbeitung}
  \begin{itemize}
  \item Alle Clojure-Datenstrukturen implementieren Seq-Interface
  \item Lazy Verarbeitung von Streams
  \end{itemize}
  \begin{block}{}
\begin{verbatim}
(map inc [1 2 3])         ; => [4 5 6]

(for [[k v] walter]
  (str (name k) ": " v))  ; => ("first-name: Walter"
                                "familiy-name: White")

\end{verbatim}
  \end{block}
\end{frame}

\begin{frame}[fragile]{Stream-Verarbeitung (2)}
  \begin{block}{}
\begin{verbatim}
(def cart-items
  [{:title "Bread",    :available? true,  :price 3}
   {:title "Bananas",  :available? false, :price 2}
   {:title "Notebook", :available? true,  :price 499}])

(->> cart-items
     (filter :available?)
     (map :price)
     (reduce +))                 ; => 502

\end{verbatim}
  \end{block}
\end{frame}

\begin{frame}[fragile]{State in Clojure}
  \begin{itemize}
  \item ``Clojure ist kein Haskell''
  \item State ist möglich - jede Funktion darf alles mutierbare mutieren
    \begin{itemize}
    \item Z.B. auch per Java-Interop
    \end{itemize}
  \item Verschiedene Arten State zu verwalten
  \item klare Multithreading-Semantik
  \end{itemize}
  \begin{block}{Atoms - synchroner Zugriff mit atomaren Updates}
\begin{verbatim}
(def state (atom 3))

(println @state)         ; => 3

(swap! state + 4)

(println @state)         ; => 7
\end{verbatim}
  \end{block}
\end{frame}

\begin{frame}{Weitere Referenztypen}
  \begin{itemize}
  \item Agents
    \begin{itemize}
    \item asynchroner Zugriff
    \end{itemize}
  \item Refs
    \begin{itemize}
    \item Software Transactional Memory
    \item Mehrere refs können transaktional geupdated werden
    \item Aber: Meist reichen Atoms
    \end{itemize}
  \end{itemize}
\end{frame}

\section{Nebenläufigkeit mit core.async}

\begin{frame}{core.async}
  \begin{itemize}
  \item core.async
    \begin{itemize}
    \item Library für asynchrone Programmierung
    \item Communicating Sequential Processes (Hoare)
    \item Von der Sprache Go kopiert (und verbessert)
    \end{itemize}
  \item Channels
    \begin{itemize}
    \item Threads können in channels schreiben oder lesen
    \item Blockieren bis read und write da sind
    \item Buffer sind möglich
    \end{itemize}
  \item go-Blöcke
    \begin{itemize}
    \item Erzeugen neuen (lightweight) Thread
    \item Nötig um auf Channels zuzugreifen
    \end{itemize}
  \end{itemize}
\end{frame}

\begin{frame}[fragile]{core.async Beispiel (1)}
  \begin{block}{}
\begin{verbatim}
(defn consumer [ch]
  (go
    (loop
      (println (<! ch))
      (recur))))


(defn producer [ch]
  (go
    (loop [i 1]
      (>! ch i)
      (recur (inc i)))))
\end{verbatim}
  \end{block}
\end{frame}

\begin{frame}[fragile]{core.async Beispiel (2)}
  \begin{block}{}
\begin{verbatim}
(defn run []
  (let [ch (chan 10)]     ; Buffer size 10
    (consumer ch)
    (producer ch)))


; =>
; 1
; 2
; 3
; 4
; ...
\end{verbatim}
  \end{block}
\end{frame}

\begin{frame}[fragile]{Wie ist core.async implementiert?}
  \begin{itemize}
  \item \lstinline{go} ist ein neues Sprachkonstrukt
  \item Code wird in eine State Machine transformiert
  \item Dennoch ist core.async nur eine Library
  \item Wie ist das möglich?
  \end{itemize}
\end{frame}

\begin{frame}[fragile]{Makros}
  \begin{itemize}
  \item Wie alle Lisps besitzt Clojure Makros
  \item Ähnlich C-Präprozessor \lstinline{#define}
  \item Aber:
    \begin{itemize}
    \item Dank einheitlicher Syntax weniger problemanfällig
    \item Makros können kompletten Clojure-Sprachumfang nutzen
    \item Metaprogrammierung ist nur Verarbeitung von Listen
    \end{itemize}
  \item Im Normalfall benötigt man aber keine Makros
    \begin{itemize}
    \item Meist overkill
    \item Funktionen sind besser komponierbar
    \item Nur wenn man wirklich die Programmiersprache erweitern möchte
    \end{itemize}
  \end{itemize}
\end{frame}

\section{Reaktive UI Programmierung mit ClojureScript}

\begin{frame}{Reaktive UI-Programmierung}
  \begin{itemize}
  \item Meine derzeitige Lieblingsanwendung von Clojure
  \item Ideal für Single-Page Web Apps
  \end{itemize}
\end{frame}

\begin{frame}{React und Reagent}
  \begin{itemize}
  \item React
    \begin{itemize}
    \item JavaScript-Library von Facebook
    \item übernimmt State-View-Synchronisierung
    \item Vorteil: Man legt nur einmal fest wie State gerendert wird
    \item HTML output kann in JS per JSX eingebettet
    \item Unter der Haube: Virtual Dom für effiziente Updates
    \end{itemize}
  \item reagent
    \begin{itemize}
    \item Eleganter ClojureScript-Wrapper um React
    \item Components sind einfach Funktionen
    \item HTML wird einfach als Vektoren repräsentiert
    \end{itemize}
  \end{itemize}
\end{frame}

\begin{frame}{Demo: Reagent}
  Zeit für eine Live-Demo!
\end{frame}

\begin{frame}{Schluss}
  \begin{itemize}
  \item Vielen Dank für die Aufmerksamkeit!
  \item Zeit für Fragen
  \end{itemize}
\end{frame}

\end{document}
